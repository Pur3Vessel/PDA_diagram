\documentclass{article}

% Language setting
% Replace `english' with e.g. `spanish' to change the document language
\usepackage[english, russian]{babel}

% Set page size and margins
% Replace `letterpaper' with `a4paper' for UK/EU standard size
\usepackage[letterpaper,top=2cm,bottom=2cm,left=3cm,right=3cm,marginparwidth=1.75cm]{geometry}

% Useful packages
\usepackage{amsmath}
\usepackage{graphicx}
\usepackage{indentfirst}
\usepackage{listings}
\usepackage{float}
\usepackage[colorlinks=true, allcolors=blue]{hyperref}

\title{Отчет по лабораторной работе №5 \\Диаграмма PDA (Вариант 3)}
\author{Баев Д.А. Федоров А.Н.}

\begin{document}
\maketitle

%\begin{abstract}
%Your abstract.
%\end{abstract}

\section{Задание}
\begin{itemize}
  \item Предложить грамматику PDA.
  \item Прочить описание PDA в заданном синтаксисе
  \item Построить диаграмму PDA, где будут выделены состояния-ловушки, недетерминированные переходы и переходы, не влияющие на стек
\end{itemize}


\section{Установка и запуск}

\begin{enumerate}
    \item Установка осуществяется путем клонируем репозиторий проекта
    \begin{lstlisting}
    git clone https://github.com/Pur3Vessel/PDA_diagram
    \end{lstlisting}
    \item Переходим в папку проекта
    \begin{lstlisting}
    cd PDA-diagram
    \end{lstlisting}
    \item Запускаем программу из корневой(!важно!) папки проекта, с указанием пути до файла с тестом и конфигом (опционально)
    \begin{lstlisting}
    lua src/main.lua path/to/test.txt path/to/config.txt
    \end{lstlisting}
    \item Построенная диаграмма автоматически выведется на экран и сохранится в файле test.pdf в корне проекта
    \begin{lstlisting}
    test.pdf
    \end{lstlisting}
    \textbf{Интерпретация диаграммы:}
    \begin{itemize}
        \item Зеленый круг (овал) - стартовое состояние
        \item Двойной круг - завершаюшее состояние
        \item Красный квадрат - ловушка
        \item Тонкая стрелка - недетерминированный переход
        \item Зеленая стрелка - стэконезависимый переход
    \end{itemize}
    
\end{enumerate}

\section{Грамматика}

\begin{figure}[H]
Стартовый нетерминал: [automata]
\centering
        $$\begin{array}{l}
        [automata] \to [initail-state],[states][sep][transitions] \;|\; [initial-state][sep][transitions]\\  
        
        [initail-state] \to [state-name] \;|\; [state-name]-[flag] \\

        [states] \to [state],[states] \;|\; [state]  \\

        [state] \to [state-name]\;|\;[state-name]-[flag] \\

        [state-name] \to [symbol] \;|\; [symbol][state-name] \\

        [transitions] \to [transiton] \;|\; [transiton][sep][transitions] \\
        
        [transition] \to [before][trans-sep][after] \;|\; \\
                        
        \qquad\qquad\qquad \ \ [state-name],[alphabet-symbol],[stack-any][trans-sep][state-name],[stack-any] \\
        
        [transition] \to [state-name],[alphabet-symbol],[stack-bottom][trans-sep][state-name],\\
        \qquad\qquad\qquad\ \ [stack-symbols][stack-bottom] \\
        
        [before] \to [state-name], [alphabet-symbol], [stack-symbol] \\
        
        [after] \to [state-name], [stack-symbols] \;|\; [state-name], [empty] \\
        
        [alphabet-symbol] \to [al-sym] \;|\; [empty] \\
        
        [stack-symbols] \to [stack-symbol][stack-symbols] \;|\; [stack-symbol] \\
        \end{array}$$
        
[symbol] - символ в имени состояния. Может быть любым разрешенным символом.

\end{figure}

\begin{itemize}
\item [al-sym] - символ входного алфавита. Задается в config некоторой регуляркой (по умолчанию: [a-z])
\item [stack-symbol] - символ стэкового алфавита. Задается в config некоторой регуляркой (по умолчанию: [A-Y])
\item [flag] - строка, которая задает метку завершающего состояния. По умолчанию: final.
\item [sep] - символ, который задает разделитель между "строками". По умолчанию: ;
\item [trans-sep] - строка, которая задает разделитель между двумя частями правила. По умолчанию: ->
\item [empty] - символ, который задает пустую строку. По умолчанию: ɛ
\item [stack-any] - символ, который задает любой стэковый символ. По умолчанию: *
\item [stack-bottom] - символ, который задает дно стэка. Не должен распознаваться регуляркой, задающей стэковый алфафит. По умолчанию: Z

\end{itemize}

\newline
\textbf{Ограничения:}

\begin{itemize}
\item Имена состояний, которые записываются в описании переходов, должны быть объявлены в [states]
\item В [symbol] запрещено использовать символы токенов, а также символы , и - и пробельные символы
\end{itemize}

\section{Заполнение config}
Параметры указываются через запятую в формате: [имя параметра] значение \\

Пример: [trans-sep] ::=, [empty] ! \\

В случае, если имя параметра указано неверно, этот параметр не будет изменен. \\

\textbf{Ограничения:}

\begin{itemize}

\item Нельзя параметризовать токены запятой и тире
\item Для избежания неоднозначности все токены должны быть уникальными значениями
\item Нельзя параметризировать токены пробельными символами
\item Стэковый алфафит, входящий алфавит и дно стэка должны параметризоваться символами ASCII
\end{itemize}

\end{document}